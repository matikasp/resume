\documentclass[a4paper,10pt]{article}
\usepackage{geometry}
\usepackage{hyperref}
\usepackage[utf8]{inputenc}
\usepackage[T1]{fontenc}
\usepackage{parskip} % better spacing between paragraphs
\usepackage{enumitem} % better list control
\usepackage{multicol} % multiple columns if needed
\usepackage{tabularx} % table alignment

\geometry{margin=0.7in}
\setlist[itemize]{left=0pt, noitemsep, topsep=2pt} % tighter item spacing
\begin{document}

\begin{center}
    {\LARGE \textbf{Mateusz Kasprzak}} \\[2pt]
    \href{mailto:mateuszkasprzak8@gmail.com}{mateuszkasprzak8@gmail.com} \,|\, 
    +48 517 605 300 \,|\, 
    \href{https://github.com/matikasp}{github.com/matikasp} \,|\, 
    \href{https://linkedin.com/in/mmkasprzak}{linkedin.com/in/mmkasprzak}
\end{center}
\vspace{-6pt}


\section*{O mnie}
Jestem studentem informatyki na Uniwersytecie Warszawskim, ukończyłem już drugi rok studiów. Chętnie zdobywam nowe doświadczenie zawodowe i stosuję moją wiedzę z programowania, algorytmów i systemów w praktycznych zastosowaniach. Lubię zarówno praktyczne tworzenie oprogramowania, jak i teoretyczną stronę informatyki. Jestem osobą szybko uczącą się i adaptującą do nowych sytuacji.

\section*{Wykształcenie}
\textbf{Uniwersytet Warszawski} \hfill 2023 -- obecnie \\
\emph{Ukończony drugi rok -- Informatyka}

\section*{Doświadczenie zawodowe}
\textbf{Stażysta Backend Software Engineering} — Kalepa \hfill lip 2025 -- wrz 2025
\begin{itemize}[left=0pt, noitemsep]
    \item Trzymiesięczny staż w startupie z branży insurance-tech.
    \item Praca nad automatyzacją i przetwarzaniem danych zawierających nieustrukturyzowane informacje.
    \item Udział w tworzeniu narzędzi wewnętrznych wspierających obsługę informacji na dużą skalę.
\end{itemize}

\section*{Projekty}
\textbf{Ewaluator wyrażeń logicznych} \hfill (Programowanie współbieżne)
\begin{itemize}[left=0pt, noitemsep]
    \item Implementacja parsera wyrażeń logicznych i współbieżnego ewaluatora obsługującego ewaluację wsadową i testy jednostkowe.
    \item Nacisk na bezpieczeństwo wątków i wydajność.
\end{itemize}

\textbf{Optymalizacja współbieżna} \hfill (Programowanie współbieżne)
\begin{itemize}[left=0pt, noitemsep]
    \item Implementacja i dostrajanie współbieżnych algorytmów/struktur danych w celu zmniejszenia konfliktów, poprawy przepustowości równoległej i skalowalności.
    \item Optymalizacje oparte na profilowaniu i benchmarking popraw skalowalności.
\end{itemize}

\textbf{Ewaluator NAND / Symulator logiczny} \hfill (Architektura komputerów)
\begin{itemize}[left=0pt, noitemsep]
    \item Budowa ewaluatora/symulatora logicznego opartego na bramkach NAND do modelowania i weryfikacji obwodów kombinacyjnych i tablic prawdy.
    \item Demonstracja programowania niskopoziomowego i zrozumienia logiki cyfrowej.
\end{itemize}

% Link to repo
\small\textit{Więcej projektów i kodu:} \href{https://github.com/matikasp/University_Projects}{github.com/matikasp/University\_Projects}

\section*{Umiejętności}
\begin{itemize}[left=0pt]
    \item \textbf{Języki programowania:} C/C++, Java, Python, NASM x86 Assembly
    \item \textbf{Oprogramowanie i narzędzia:} Git, Linux, AWS, Flask, OpenAI API
    \item \textbf{Systemy:} Systemy operacyjne, architektura komputerów, sieci komputerowe, programowanie współbieżne
    \item \textbf{Informatyka teoretyczna:} Algorytmy i struktury danych, teoria automatów, języki formalne, złożoność obliczeniowa
    \item \textbf{Uczenie maszynowe:} Podstawowa znajomość koncepcji ML i bibliotek (np. scikit-learn, NumPy, pandas)
\end{itemize}

\section*{Języki}
\begin{itemize}[left=0pt]
    \item \textbf{Polski:} Ojczysty
    \item \textbf{Angielski:} Płynny, poziom B2 (samoocena)
\end{itemize}

\end{document}